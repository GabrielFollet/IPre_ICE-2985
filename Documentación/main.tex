\documentclass[a4paper, 11pt,article,oneside]{memoir}%,openany
\setlrmarginsandblock{2.5cm}{2.5cm}{*}
\setulmarginsandblock{2.5cm}{2.5cm}{*}
\checkandfixthelayout
\usepackage[utf8]{inputenc} 
\usepackage[T1]{fontenc} 
\usepackage{pifont}
\usepackage{amsmath}
\usepackage[dvipsnames]{xcolor}
\usepackage[breakable]{tcolorbox}
    \tcbuselibrary{skins}
\usepackage[spanish]{babel}
\usepackage{stix}
\usepackage{graphicx}
\usepackage[final]{hyperref}
\usepackage{listings}
\usepackage{float}
\usepackage{soul}
\usepackage{subcaption}
\usepackage{graphicx}
%===================================================================%
\usepackage{listings}
% This is the color used for MATLAB comments below
\definecolor{MyDarkGreen}{rgb}{0.0,0.4,0.0}

% For faster processing, load Matlab syntax for listings
\lstloadlanguages{Matlab}%
\lstset{language=Matlab,                        % Use MATLAB
        frame=single,                           % Single frame around code
        basicstyle=\small\ttfamily,             % Use small true type font
        keywordstyle=[1]\color{Blue}\bfseries,        % MATLAB functions bold and blue
        keywordstyle=[2]\color{Purple},         % MATLAB function arguments purple
        keywordstyle=[3]\color{Blue}\underbar,  % User functions underlined and blue
        identifierstyle=,                       % Nothing special about identifiers
                                                % Comments small dark green courier
        commentstyle=\usefont{T1}{pcr}{m}{sl}\color{MyDarkGreen}\small,
        stringstyle=\color{Purple},             % Strings are purple
        showstringspaces=false,                 % Don't put marks in string spaces
        tabsize=5,                              % 5 spaces per tab
        %
        %%% Put standard MATLAB functions not included in the default
        %%% language here
        morekeywords={xlim,ylim,var,alpha,factorial,poissrnd,normpdf,normcdf},
        %
        %%% Put MATLAB function parameters here
        morekeywords=[2]{on, off, interp},
        %
        %%% Put user defined functions here
        morekeywords=[3]{FindESS, homework_example},
        %
        %morecomment=[l][\color{Blue}]{...},     % Line continuation (...) like blue comment
      % numbers=left,                           % Line numbers on left
      %  firstnumber=1,                          % Line numbers start with line 1
       % numberstyle=\tiny\color{blue},          % Line numbers are blue
       % stepnumber=1                            % Line numbers go in steps of 1
        }
        
       
    \newenvironment{myitemize}{%
\begin{itemize}}{\end{itemize}}
\tcolorboxenvironment{myitemize}{blanker,
before skip=4pt,after skip=4pt,
borderline west={  0.8mm}{0pt}{lightgray}}

 \newenvironment{myitemize2}{%
\begin{itemize}}{\end{itemize}}
\tcolorboxenvironment{myitemize2}{ breakable,blanker,
before skip=10pt,after skip=10pt,
borderline west={  0.8mm}{0pt}{black}}

 \newenvironment{myitemize3}{%
\begin{itemize}}{\end{itemize}}
\tcolorboxenvironment{myitemize3}{blanker,
before skip=10pt,after skip=10pt,
borderline west={  0.8mm}{0pt}{cyan}}
\begin{document}
%	TITLE PAGE
\begin{titlingpage}
	\raggedleft % ]
	\rule{3 pt}{\textheight} % Vertical line
	\hspace{0.05\textwidth} % Whitespace between the vertical line and title page text
	\parbox[b]{0.9\textwidth}{ % Paragraph box for holding the title page text, adjust the width to move the title page left or right on the page
		
		{\\\Huge\bfseries Manual de Apoyo al 	\\[0.5\baselineskip] Análisis de Ensayos Cíclicos de  \\[0.5\baselineskip]  Muros Hormigón Armado}	\\[2\baselineskip] % Title
		{\large\textit{ Script de  Matlab \textit{Analisis_Data_Experimental.m}}}\\[4\baselineskip] % Subtitle or further description
		{\Large\textsc{Gabriel Follet}} % Author name, lower case for consistent small caps\\
      	\\[4\baselineskip] 
		
		
		\vspace{0.30\textheight} % Whitespace between the title block and the publisher
		
		{\noindent Pontificia Universidad Católica de Chile\\Departamento de Ingeniería Estructural y Geotécnica\\
		gabriel.follet@uc.cl\\
		v3-diciembre 2022} % Publisher and logo	
		}

\end{titlingpage}
\setsecindent{2\parindent}
\setsubsecindent{0.5\parindent}
\setsecheadstyle{\scshape\raggedright}
\setsubsecheadstyle{\scshape\raggedright}
\setsubsubsecheadstyle{\scshape\raggedright}
\chapterstyle{bringhurst}
\tableofcontents
\newpage
\listoffigures
\newpage
\chapter{Introducción}
Este manual funciona como un apoyo al script \\ \textit{Analisis\_Data\_Experimental.m} desarrollado para analizar ensayos cíclicos reversibles a deformación controlada. \par
El script ajusta un backbone y un modelo bilineal a la data experimental. También calcula múltiples rigideces, energía disipada y ductilidad, entre otros parámetros.
La data experimental del ensayo debe de estar ``relativamente limpia", en específico, \textbf{no} debe que poseer: 
\begin{itemize}
    \item Pausas iniciales.
    \item Pausas finales.
    \item Pausas intermedias donde la deformación oscila entorno al cero. 
\end{itemize}
El script  \textit{Analisis\_Data\_Experimental.m} y las funciones auxiliares del Anexo A, se encuentran disponibles en \url{https://github.com/GabrielFollet/IPre_ICE-2985} 
\newpage
\chapter{Carga de Data Experimental}\label{sec:Inputs}
Es necesario cargar dentro del Script  6 variables. En la figura \ref{fig:cargadata} se muestra a sección de carga de data dentro del Script.\par
 \begin{figure} [h!]
    \centering
    \includegraphics[width=1\textwidth]{Figuras/Figuras Introducción/carga data.PNG}
    \caption{\label{fig:cargadata} Carga de data en Script}
\end{figure}
El formato y significado de cada una de estas variables es :
\begin{itemize}
\item \textit{Raw Data} :\label{subsec:RawData}
Corresponde a la data experimental del ensayo. La estructura de esta data debe de ser  siguiente 
    \begin{itemize}
        \item $RawData(:,1) = $ Vector columna con los data de la  deformación.
        \item $RawData(:,2) = $ Vector columna con la data de la fuerza.
    \end{itemize}
\item\textit{Id}: Corresponde a un \href{https://www.mathworks.com/help/matlab/ref/string.html?searchHighlight=strings&s_tid=srchtitle_strings_1}{string array} con la identificación del muro ensayado.
\item\textit{Fuente}: 
 Corresponde a un string con la fuente de la data experimental.
\item \textit{Lv}: Corresponde a un float con el ShearSpan del muro, es decir con al altura desde la base del muro al actuador.
\item \textit{Unidades} : Corresponde a un vector fila. El primer elemento corresponde a un string con la unidad de la deformación y el segundo elemento a un string con la unidad de fuerza. Es importante recalcar que las unidad de la deformación debe de ser consistente con la unidad de \textit{Lv}.
\item{\textit{Comentarios}}: Esta variables es un string que se utiliza parar almacenar comentarios relacionados a la data experimental.
\end{itemize}
\newpage
\chapter{Outputs}
La rutina tiene 3 \textit{outputs}. El primero, consiste en la impresión en la consola de un  resumen de lo realizado por el Script, el segundo corresponde a gráficas útiles para validar gráficamente lo realizado por el Script, finalmente un   \href{https://www.mathworks.com/help/matlab/ref/struct.html}{structure array} llamado \textit{DataProcesada} compuesto por 11 atributos. En la figuras \ref{fig:first resumen}, \ref{fig:Validación Visual} y \ref{fig:atributes } se muestra un ejemplo de los \textit{outputs} del script.
\begin{figure}[h!]
    \centering
    \includegraphics[width=\textwidth]{Figuras/Consola/consola.PNG}
    \caption{Impresión de Resumen en la consola}
    \label{fig:first resumen}
\end{figure}





\begin{figure}[h!]
    \centering
    \begin{subfigure}{0.45\textwidth}
   \includegraphics[width=1\textwidth]{Figuras/Consola/Validación 1.png}
    \caption{Validación visual: Detección de ciclo}
    \label{fig:Validación 1}
\end{subfigure}
\hfill
\begin{subfigure}{0.45\textwidth}
   \includegraphics[width=1\textwidth]{Figuras/Figuras Introducción/Resuemne_total.png}
    \caption{Validación Visual: Modelo Ajustado}
    \label{fig:second}
\end{subfigure}
\caption{ Gráficas Validación Visual}
\label{fig:Validación Visual}
\end{figure}
\begin{figure}[h!]
    \centering
    \includegraphics[width=0.3\textwidth]{Figuras/Figuras Introducción/atributos.PNG}
    \caption{struct \textit{DataProcesada} con sus atributos}
    \label{fig:atributes }
\end{figure}
En los siguientes capítulos se explica el contenido del struct \textit{DataProcesada}
 \newpage
\chapter{Perfiles y Opciones manuales}
El script utiliza múltiples variables para ajustar el funcionamiento. Definiéndose 11 variables, que pueden ser modificadas en la sección \textit{ Perfiles y Parámetros opcionales}. Para facilitar el funcionamiento se definieron 2 perfiles que establecen los valores de estas variables, como se observa en la figura \ref{fig:perfeeee}.
A continuación se mencionar brevemente el significado de cada perfil.
\begin{itemize}
    \item Perfil=1\\
    Perfil que  corresponde a ajustar un Backbone considerando el punto de deformación máxima del primer ciclo a cada nivel de deformación, y ocupa este backbone para ajustar un modelo bi-lineal con rigidez post-fluencia.
    \item Perfil=2\\
   Perfil que corresponde a ajustar un Backbone considerando el punto de deformación máxima del primer ciclo a cada nivel de deformación, y ocupar este backbone para ajustar un modelo bi-lineal elasto-plástico.
\end{itemize}
A continuación se muestra la sección del código donde están todas las variables a modificar, con una breve descripción, en el presente manual se explicarán con mayor detalle  y se marcaran con \textit{\textcolor{blue}{color azul}}, para facilitar la compresión.
 \begin{figure}[H]
    \centering
    \includegraphics[width=1\textwidth]{Figuras/parametro_opcionales}
    \caption{Algunas opciones dentro del Script}
    \label{fig:perfeeee}
\end{figure}
\newpage
\chapter{Data.Procesada.Data}
 El atributo \textit{Data.Procesada.Data} es a su vez un struct con 2 atributos, según lo que se muestra en la figura \ref{fig:.datab }.\par
\begin{figure}[h!]
    \centering
    \includegraphics[width=0.4\textwidth]{Figuras/.data.PNG}
    \caption{struct \textit{DataProcesada.Data} con sus atributos}
    \label{fig:.datab }
\end{figure}
A continuación se explican el contenido de cada uno de los atributos 
\section{DataProcesada.Data.RawData}
 Corresponde a un array con la data cargada originalmente
\section{DataProcesada.Data.ProcessedData}
Corresponde a un array con la data procesada. El procesamiento consisten en: 
\begin{itemize}
    \item Eliminación de \textit{Ouliers}
    \item Eliminación de puntos iniciales
    \item Interpolación de puntos
\end{itemize}
\subsection{Eliminación de Outliers}
La detección de \textit{outlier} se realiza con la función interna de Matlab \href{https://www.mathworks.com/help/matlab/ref/rmoutliers.html?s_tid=doc_ta}{\textit{rmoutliers}}, particularmente se decidió utilizar el método  \textit{movmedian}, considerando una ventada de tamaño \textit{\textcolor{blue}{window}} para el cálculo de la mediana movil.
Queda a discreción del usuario la utilización de este método o  utilizar otros de los que \href{https://www.mathworks.com/help/matlab/ref/cleanoutlierdata.html}{Matlab posee}. 
\subsection{Eliminación de puntos iniciales}
Si bien se presupone que la data experimental está ``relativamente limpia'', se realiza una pequeña limpieza de la data que consiste en :
\begin{itemize}
\item Eliminar todos los puntos iniciales que tiene deformación negativa o fuerza negativa. 
\item Insertar el punto $(0,0)$ en la primera posición.
\end{itemize}
\subsection{Interpolación de puntos}\label{subsec: interp}
Se define que un ciclo comienza cuando la deformación pasa de negativa a positiva. Es así, que el Script interpola linealmente entre estos dos puntos agregando el punto $(0,f)$ el que marca el fin del ciclo anterior y el inicio de un nuevo ciclo. También, se interpolan los puntos $(d^+,0) \quad || (d^-,0)$, que definen la parte positiva y negativa de cada ciclo.
    A continuación se presenta un ejemplo del resultado de estas interpolaciones
     \begin{figure} [H]
    \centering
    \includegraphics[width=1\textwidth]{Figuras/figuras interpolacion/ejemplo de puntos interpolado.png}
    \caption{\label{interp result} Resultado de interpolación de puntos para un ciclo}
    \end{figure}
\newpage
\chapter{Data.Procesada.Ciclos}
El atributo \textit{Data.Procesada.Ciclos} es a su vez un struct con 3 atributos, según lo que se muestra en la siguiente figura.\\

\begin{figure}[H]
    \centering
    \includegraphics[width=0.4\textwidth]{Figuras/.ciclos.PNG}
    \caption{struct \textit{DataProcesada.Ciclos} con sus atributos}
    \label{fig:first resumen e2 }
\end{figure}

A continuación, se explica el contenido de cada uno de los atributos 
\section{Data.Procesada.Ciclos.DataCiclos}
Corresponde un \href{https://www.mathworks.com/help/matlab/cell-arrays.html}{cell array}, donde cada elemento es un array con la data experimental de los puntos de los ciclos detectados.\par 
Si algún ciclo presenta insuficiente puntos muestrales, el ciclo, y todos los puntos asociados son eliminados de la data procesada. La cantidad mínima de puntos queda determinado por 
\begin{align*}
max(\textit{\textcolor{blue}{MinPointCicle}},\textit{\textcolor{blue}{Alpha}} *length(DataCiclos{1}))
\end{align*}

\section{Data.Procesada.Ciclos.puntoscriticos}
Corresponde a un vector con los puntos críticos. Se define como punto crítico a un punto donde la deformación es exactamente igual a cero. Los que corresponden a puntos que originalmente tenían deformación igual a 0 y los puntos recientemente interpolados. 
\section{Data.Procesada.Ciclos.Protocolo}
Corresponde a una \href{https://www.mathworks.com/help/matlab/ref/table.html?searchHighlight=table&s_tid=srchtitle_table_1}{tabla} con el protocolo de carga identificado. Se ajustó el protocolo considerando la (semi)amplitud y el número de ciclos a cada amplitud.\par La semi-amplitud de cada ciclo se determinó a partir de la siguiente convención
\begin{itemize}
    \item Ciclo Inicial \\
    La semi-amplitud es el mayor valor entre la deformación máxima y mínima
    \item  Ciclos intermedios\\
    El valor de la semi-amplitud es el promedio entre la deformación máxima y mínima.
    \item Ciclo Final\\
     La semi-amplitud es el mayor valor entre la deformación máxima y mínima
\end{itemize}
Esta parte de Script
    se define una variable llamada \textit{deformacion} que guarda el valor de la semi-amplitud a la cual comparar, si la semi-amplitud de un ciclo subsecuente es igual a la semi-amplitud a comparar $\pm$,  \textit{\textcolor{blue}{CicleTolerance}}, se considera que este ciclo pertenece al grupo de ciclos de igual deformación. Este valor se  actualiza, con  el valor del promedio de todos los ciclos que pertenecen al grupo.\par
    El intervalo que define si un ciclo pertenece a un grupo, queda definido entonces por
    \begin{equation*}
         (\textit{deformacion}*(1-\textit{\textcolor{blue}{CicleTolerance}},)\, ,\, \textit{deformacion}*(1+\textit{\textcolor{blue}{CicleTolerance}},)
    \end{equation*}
    \newpage
\chapter{DataProcesada.Deriva}
El atributo \textit{DataProcesada.Deriva} corresponde a su vez a un struct con dos atributos como se muestra en la siguiente figura
\begin{figure}[h!]
    \centering
    \includegraphics[width=0.4\textwidth]{Figuras/.deriva.PNG}
    \caption{struct \textit{DataProcesada.Deriva} con sus atributos}
    \label{fig:first resumen e3 }
\end{figure}.
\section{DataProcesada.Deriva.MaxDeriva}
Corresponde a un array con la deriva máxima alcanzada en cada ciclo
\section{DataProcesada.Deriva.MinDeriva}
Corresponde a un array con la deriva mínima alcanzada en cada ciclo
\newpage
\chapter{DataProcesada.Fuerza}
El atributo \textit{DataProcesada.Fuerza} corresponde a su vez a un struct con dos atributos como se muestra en la siguiente figura
\begin{figure}[h!]
    \centering
    \includegraphics[width=0.4\textwidth]{Figuras/.fuerza.PNG}
    \caption{struct \textit{DataProcesada.Fuerza} con sus atributos}
    \label{fig:first resumen e3 }
\end{figure}.
\section{DataProcesada.Fuerza.MaxFuerza}
Corresponde a un array con la fuerza máxima alcanzada en cada ciclo
\section{DataProcesada.Fuerza.MinFuerz}
Corresponde a un array con la fuerza mínima alcanzada en cada ciclo
\newpage
\chapter{DataProcesada.Rigidez}
El atributo \textit{DataProcesada.Rigidez} corresponde a su vez a un struct con tres atributos como se muestra en la siguiente figura.\par
\begin{figure}[h!]
    \centering
    \includegraphics[width=0.4\textwidth]{Figuras/.rigidez.PNG}
    \caption{struct \textit{DataProcesada.Rigidez} con sus atributos}
    \label{fig:first resumen e4 }
\end{figure}
Se consideraron 2 maneras de estimar  la rigidez dentro de un ciclo
\subsection{Método 1}\label{subsec:metodo 1}
Considera que la rigidez queda determinado por el punto de fuerza máxima. \par
Se calculan las siguientes rigideces con este método
\begin{itemize}
    \item Rigidez positiva de Carga ( $K_{c1}$)\\
    Corresponde a la rigidez determinado por los puntos
    $$ (D_i,F=0)\quad || \quad (D,F_{max})$$
    Donde $D_i$ corresponde al punto donde la fuerza es nula en carga.
    \item Rigidez Positiva de Descarga ($K_{d1}$)\\
     Corresponde a la rigidez determinado por los puntos
    $$ \quad (D,F_{max})\quad || \quad (D_j,F=0))$$
    Donde $D_j$ corresponde al punto donde la fuerza es nula en descarga.
    \item Rigidez Negativa de Carga ( $K^{-}_{c1}$)\\
    Corresponde a la rigidez determinado por los puntos
    $$ (D_j,F=0)\quad || \quad (D,F_{min})$$
    Donde $D_j$ corresponde al punto donde la fuerza es nula en carga.
     \item Rigidez Negativa de Descarga ($K^-_{d1}$)\\
     Corresponde a la rigidez determinado por los puntos
    $$ \quad (D,F_{min})\quad || \quad (D_i,F=0))$$
    Donde $D_i$ corresponde al punto donde la fuerza es nula en descarga.
    \item Rigidez Secante($K_{s1}$)
    Corresponde a la rigidez determinado por los puntos
    $$ \quad (D,F_{min})\quad || \quad (D,F_{max}))$$
    \end{itemize}
\subsection{Método 2}\label{subsec: metodo 2}
Considera que la rigidez queda determinado por el punto de deformación máxima. \par 
Se calculan las siguientes rigideces con este método
\begin{itemize}
    \item Rigidez positiva de Carga ( $K_{c2}$)\\
    Corresponde a la rigidez determinado por los puntos
    $$ (D_i,F=0)\quad || \quad (D_{max},F)$$
    Donde $D_i$ corresponde al punto donde la fuerza es nula en carga.
    \item Rigidez Positiva de Descarga ($K_{d2}$)\\
     Corresponde a la rigidez determinado por los puntos
    $$ \quad (D_{max},F)\quad || \quad (D_j,F=0)$$
    Donde $D_j$ corresponde al punto donde la fuerza es nula en descarga.
    \item Rigidez Negativa de Carga ( $K^{-}_{c2}$)\\
    Corresponde a la rigidez determinado por los puntos
    $$ (D_j,F=0)\quad || \quad (D_{min},F)$$
    Donde $D_j$ corresponde al punto donde la fuerza es nula en carga.
     \item Rigidez Negativa de Descarga ($K^-_{d2}$)\\
     Corresponde a la rigidez determinado por los puntos
    $$ \quad (D_{min},F)\quad || \quad (D_i,F=0)$$
    Donde $D_i$ corresponde al punto donde la fuerza es nula en descarga.
    \item Rigidez Secante($K_{s2}$)
    Corresponde a la rigidez determinado por los puntos
    $$ \quad (D_{min},F)\quad || \quad (D_{max},F)$$
    \end{itemize}
 A continuación se muestra gráficamente las rigideces calculadas en cada ciclo.
    \begin{figure}[h!]
    \centering
    \begin{subfigure}{1\textwidth}
    \includegraphics[width=\textwidth]{Figuras/Rigideces/ejemplo rigideces.png}
    \caption{Rigideces parte positiva}
    \label{fig:Rigidez positica}
\end{subfigure}
\hfill
\begin{subfigure}{1\textwidth}
    \includegraphics[width=\textwidth]{Figuras/Rigideces/rigidez neg.png}
    \caption{Rigideces parte negativa}
    \label{fig:Rigidez negativa}
\end{subfigure}
\caption{Estimación de rigideces en un ciclo}
\label{fig:figures}
\end{figure}
\section{DataProcesada.Rigidez.Carga}
Corresponde a un array donde: 
\begin{itemize}
    \item La primera fila es la rigidez de carga por el método 1 en la parte positiva en cada ciclo.
    \item  La segunda fila es la rigidez de carga por el método 2 en la parte positiva en cada ciclo.
     \item  La tercera fila es la rigidez de carga por el método 1 en la parte negativa en cada ciclo.
     \item  La cuarta fila es la rigidez de carga por el método 2 en la parte negativa en cada ciclo.
\end{itemize}
\section{DataProcesada.Rigidez.Descarga}
Corresponde a un array donde: 
\begin{itemize}
    \item La primera fila es la rigidez de descarga por el método 1 en la parte positiva en cada ciclo.
    \item  La segunda fila es la rigidez de descarga por el método 2 en la parte positiva en cada ciclo.
     \item  La tercera fila es la rigidez de descarga por el método 1 en la parte negativa en cada ciclo.
     \item  La cuarta fila es la rigidez de descarga por el método 2 en la parte negativa en cada ciclo.
\end{itemize}
\section{DataProcesada.Rigidez.Secante}
Corresponde a un array donde: 
\begin{itemize}
    \item La primera fila es la rigidez de secante por el método 1.
    \item  La segunda fila es la rigidez de secante por el método 2.
\end{itemize}
\newpage
\section{Rigideces en último ciclo}
Las convenciones explicada anteriormente para estimar la rigidez de un ciclo, pueden no ser adecuadas si el ciclo final no se completo, es decir, si el ultimo  punto experimental no es $(0,0)$. De esta manera se consideró por defecto que la rigidez del último ciclo ( para todas las posibles tipos ) es nula, y se establecieron condiciones para verificar que las convenciones de rigidez definidas anteriormente sean aplicables y representativos del ciclo. Las condiciones para cada "tipo de rigidez" son
\begin{itemize}
    \item \textsc{.Rigidez.Carga(1,:)}\\
    Se considera que el método es aplicable si 
    \begin{align*}
        \textit{\textcolor{blue}{PartialCicleTolerance}}*maxf(end-1)<maxf(end)
    \end{align*}
    \item \textsc{.Rigidez.Carga(2,:)}\\
    Se considera que el método es aplicable si 
    \begin{align*}
        \textit{\textcolor{blue}{PartialCicleTolerance}}*maxdef(end-1)<maxdef(end)
    \end{align*}
    \item \textsc{.Rigidez.Carga(3,:)} y \textsc{.Rigidez.Secante(1,:)}\\
    Se considera que el método es aplicable si existen puntos muestrales en la parte negativa de las histéresis y si 
    \begin{align*}
       \textit{\textcolor{blue}{PartialCicleTolerance}}*minf(end-1)>minf(end)
    \end{align*}
     \item \textsc{.Rigidez.Carga(4,:)} y \textsc{.Rigidez.Secante(2,:)}\\
    Se considera que el método es aplicable si existen puntos muestrales en la parte negativa de las histéresis y si 
    \begin{align*}
       \textit{\textcolor{blue}{PartialCicleTolerance}}*mindef(end-1)>mindef(end)
    \end{align*}
    \item \textsc{.Rigidez.Descarga(1,:)} y \textsc{.Rigidez.Descarga(2,:)} \\
    Se considera que el método es aplicable si
    \begin{center}
         Existe un punto con coordenadas $(d,0)$, punto que  solo existe según lo mencionado \\en \hyperref[subsec: interp]{Interpolación de Puntos}
    \end{center}
   
    O, si  "la deformación del ultimo punto experimental es menor  que una fracción de la deformación máxima del ciclo", es decir
    \begin{align*}
        Ciclos\{end\}(end,1)>\textit{\textcolor{blue}{MidPointTolerance}}*maxdef(end)
    \end{align*}
    \item \textsc{.Rigidez.Descarga(3,:)} y \textsc{.Rigidez.Descarga(4,:)} \\
    Se consideran que lo método anterior aplican si el valor absoluto de la  deformación del ultimo punto experimental es menor que una fracción del valor absoluto de la deformación mínima del ciclo", es decir
    \begin{align*}
        Ciclos\{end\}(end,1)>\textit{\textcolor{blue}{FinalPointTolerance}}*abs(mindef(end))
    \end{align*}
    
   
\end{itemize}
\newpage
\chapter{DataProcesada.Energia}
El atributo \textit{DataProcesada.Energia} corresponde a su vez a un struct con tres atributos como se muestra en la siguiente figura
\begin{figure}[h!]
    \centering
    \includegraphics[width=0.5\textwidth]{Figuras/.energia.PNG}
    \caption{struct \textit{DataProcesada.Energia} con sus atributos}
    \label{fig:first resumen e4 }
\end{figure}.
\section{DataProcesada.Energia.EnergiaDisipada}
Corresponde a un vector con la energía disipada en cada ciclo. Se calculó la energía como el área encerrada por cada ciclo como se observa en la figura \ref{fig:first resumen e543}
\begin{figure}[h!]
    \centering
    \includegraphics[width=0.7\textwidth]{Figuras/.energiaDisipada.png}
    \caption{Energía disipada en un ciclo}
    \label{fig:first resumen e543}
\end{figure}.
\section{DataProcesada.Energia.EnergiaDisipadaAcumulada}
Corresponde a un vector con la energía disipada acumulada hasta cada ciclo.
\section{DataProcesada.Energia.EnergiaDisipadaAcumuladaNormalizada}
Corresponde a un vector con la energía disipada acumulad hasta el fin de cada ciclo, normalizado con la energía disipada por el modelo bi-lineal ajustado según lo que se observa en la figura \ref{fig:first resumen e44 }

\begin{figure}[h!]
    \centering
    \includegraphics[width=0.7\textwidth]{Figuras/.energiaNorm.png}
    \caption{Energía de normalización}
    \label{fig:first resumen e44 }
\end{figure}.
\newpage\null \newpage
\chapter{DataProcesada.Info}
El atributo \textit{DataProcesada.Info} corresponde a su vez a un struct con cuatro atributos como se muestra en la siguiente figura.\par
\begin{figure}[h!]
    \centering
    \includegraphics[width=0.4\textwidth]{Figuras/.info.PNG}
    \caption{struct \textit{DataProcesada.Info} con sus atributos}
    \label{fig:first resumen e4 }
\end{figure}
\section{DataProcesada.Info.Fuente}
Corresponde a un string con la fuente de la data experimental cargada
\section{DataProcesada.Info.ShearSpan}
Corresponde a un entero con el valor del \textit{ShearSpan} ingresado en la  sección de carga de data experimental en el script.
\section{DataProcesada.Info.Unidades}
Corresponde a un array con las unidades de la data experimental cargada.
\section{DataProcesada.Info.ID}
Corresponde a un array con el ID del muro asociado a la data experimental cargada.
\newpage
\chapter{DataProcesada.PerdidaRigidez}
El atributo \textit{DataProcesada.PerdidaRigidez} corresponde a su vez a un struct con tres atributos como se muestra en la siguiente figura
\begin{figure}[h!]
    \centering
    \includegraphics[width=0.4\textwidth]{Figuras/.PerdidaRigidez.PNG}
    \caption{struct \textit{DataProcesada.PerdidaRigidez} con sus atributos}
    \label{fig:first resumen e4 }
\end{figure}.
 \section{Normalización}
 La rigidez con la cual se normalizará queda determinada por la variable  \textit{\textcolor{blue}{KNormOption}}.
\begin{itemize}
    \item \textit{\textcolor{blue}{KNormOption}}=1\\
    Se  utiliza la rigidez del tramo elástico del modelo bilineal ajustado.
    \item \textit{\textcolor{blue}{KNormOption}}=2\\
    Esta opción calcula la rigidez secante al punto de la curva esqueleto donde la fuerza es igual a $\textit{\textcolor{blue}{Q}}*F_{max}$.\par
    Por defecto el valor de $\textit{\textcolor{blue}{Q}}=0.6$. A continuación se muestra una figura con el efecto de distintos valores de \textit{\textcolor{blue}{Q}}.
    \begin{figure} [H]
    \centering
    \includegraphics[width=1\textwidth]{Figuras/Bi-Lineal/Rigidez_normalizar.png}
    \caption{\label{Bi-Lineal norm} Comparación distintos valores de \textit{\textcolor{blue}{Q}}}
    \end{figure}
\end{itemize}
Similarmente, la rigidez secante considera los siguientes puntos en la curva esqueleto 
\begin{equation*}
    (D,\textit{\textcolor{blue}{Q}}F_{min})\quad ||\quad (D,\textit{\textcolor{blue}{Q}}F_{max})
\end{equation*}
Si en cambio,se utiliza  el método \textit{\textcolor{blue}{KNormOption}}=1, la rigidez secante de normalización considera los los puntos: 
\begin{equation*}
    (D_y^-,F_y^-)\quad ||\quad (D_y^+,F_y^+)
\end{equation*}
\section{DataProcesada.PerdidaRigidez.Carga}
Corresponde a un array donde: 
\begin{itemize}
    \item La primera fila es la rigidez normalizada de carga por el método 1 en la parte positiva en cada ciclos.
    \item  La segunda fila es la rigidez normalizada de carga por el método 2 en la parte positiva en cada ciclo.
     \item  La tercera fila es la rigidez normalizada de carga por el método 1 en la parte negativa en cada ciclo.
     \item  La cuarta fila es la rigidez de carga normalizada por el método 2 en la parte negativa en cada ciclo.
\end{itemize}
\section{DataProcesada.PerdidaRigidez.Descarga}
Corresponde a un array donde: 
\begin{itemize}
    \item La primera fila es la rigidez normalizada de descarga por el método 1 en la parte positiva en cada ciclos.
    \item  La segunda fila es la rigidez normalizada de descarga por el método 2 en la parte positiva en cada ciclo.
     \item  La tercera fila es la rigidez normalizada de descarga por el método 1 en la parte negativa en cada ciclo.
     \item  La cuarta fila es la rigidez de descarga normalizada por el método 2 en la parte negativa en cada ciclo.
\end{itemize}
\section{DataProcesada.PerdidaRigidez.Secante}
Corresponde a un array donde: 
\begin{itemize}
    \item La primera fila es la rigidez normalizada de secante por el método 1.
    \item  La segunda fila es la rigidez normalizada de secante por el método 2.
\end{itemize}
\newpage
\chapter{DataProcesada.PerdidaFuerza}
El atributo \textit{DataProcesada.Rigidez} corresponde a su vez a un struct con tres atributos como se muestra en la siguiente figura.

\begin{figure}[H]
    \centering
    \includegraphics[width=0.4\textwidth]{Figuras/.PerdidaFuerza.PNG}
    \caption{struct \textit{DataProcesada.PerdidaFuerza} con sus atributos}
    \label{fig:first resumen e5 }
\end{figure}.
\section{DataProcesada.PerdidaFuerza.FuerzaMaxima}
Corresponde a un array donde cada columna representa la pérdida de fuerza a ese nivel de deformación (según el protocolo identificado). Cada elemento de un fila corresponde a la razón entre la fuerza máxima en los ciclos a ese nivel de deformación y la fuerza máxima del primer ciclo a ese nivel de deformación.
\section{DataProcesada.PerdidaFuerza.FuerzaMinima}
Corresponde a un array donde cada columna representa la pérdida de fuerza a ese nivel de deformación (según el protocolo identificado). Cada elemento de un fila corresponde a la razón entre la fuerza mínima en los ciclos a ese nivel de deformación y la fuerza mínima del primer ciclo a ese nivel de deformación.
\newpage
\chapter{DataProcesada.Backbone}
El atributo \textit{DataProcesada.Rigidez} corresponde a su vez a un struct con cinco atributos como se muestra en la siguiente figura.\par
\begin{figure}[h!]
    \centering
    \includegraphics[width=0.4\textwidth]{Figuras/backbone.PNG}
    \caption{struct \textit{DataProcesada.Backbone} con sus atributos}
    \label{fig:first resumen e6 }
\end{figure}.
\section{DataProcesada.Backbone.Backbone}.
El script tiene la capacidad de ajustas 4 tipos de curva esqueleto, el modelo ajustado dependerá del valor de la variable \textit{\textcolor{blue}{BackBoneOption}}.\par 
A continuación se explican los valores que puede tomar a variables y los modelos asociados a cada valor.
\begin{itemize}
\item \textit{\textcolor{blue}{BackBoneOption}}=1\\
Denominado como Deformación máxima\_v1. En este modelo el backbone pasa por el punto de máxima deformación del primer ciclo  en cada incremento de deformación.
\item \textit{\textcolor{blue}{BackBoneOption}}=2\\
Denominado como el modelo de Fuerza máxima.EL backbone  pasa por el punto de fuerza máxima del primer ciclo a cada incremento de deformación.
\item \textit{\textcolor{blue}{BackBoneOption}}=3\\
Denominado como el backbone de Deformación máxima \_v2. Se construye considerando que la curva esqueleto pasa por el punto de máxima deformación entre todos los ciclos a un nivel de deformación, para cada incremento de deformación.
\end{itemize}.
\section{DataProcesada.Backbone.ModeloBilineal}
Corresponde a un array con la coordenadas del modelo Bilineal ajustado. El script puede calcular dos tipos de curvas Bi-lineales, lo que depende del valor de a variable \textit{\textcolor{blue}{BiLinealOption}}
 En este modelo se iguala la energía que la bilineal sobre-estima en el tramo lineal-elástico con la energía que se subestima en el tramo post fluencia. La tolerancia del error entre estas dos áreas esta definida por la variable. \textit{\textcolor{blue}{EnergyTolerance}}.Los valores que puede tomar esta variables y los modelos que representa cada valor se explican a continuación
\begin{itemize}
    \item  \textit{\textcolor{blue}{BiLinealOption}}=1\\
    Modelo con rigidez post-fluencia, este punto de fluencia es  tal que la fuerza en la envolvente a esa deformación sea el $0.6 F_{max}$.\par A continuación se muestra una figura con este modelo.
     \begin{figure} [H]
    \centering
    \includegraphics[width=0.6\textwidth]{Figuras/Bi-Lineal/bilineal 1.PNG}
    \caption{\label{Bi-Lineal 1} Ajuste modelo bilineal 1}
    \end{figure}
    \item  \textit{\textcolor{blue}{BiLinealOption}}=2\\
    Modelo sin rigidez post-fluencia. El punto de fluencia corresponde al punto en la curva esqueleto donde la fuerza es igual al $0.75 F_{max}$. \par 
    A continuación se muestra  un ejemplo del ajuste del modelo 
    \begin{figure} [H]
    \centering
    \includegraphics[width=0.6\textwidth]{Figuras/Bi-Lineal/bilineal 2.PNG}
    \caption{\label{Bi-Lineal 2} Ajuste modelo bilineal 2}
    \end{figure}
    
\end{itemize}
\section{DataProcesada.Backbone.Falla}
El Script  detecta  dos tipos de falla del muro. 
\begin{itemize}
    \item Falla por Pérdida de Fuerza a deformación mayor.\\
    Se define que un muro falla cuando la fuerza a un nivel de deformación es inferior a la fuerza  del nivel de deformación anterior.
    Se establecieron dos posibles casos
    \begin{itemize}
        \item Ciclos intermedios\\
        Para niveles de deformación distintos al  último que alcanzó al muro. Se establece que el muro falla cuando 
        \begin{align*}
            Fuerza_{i}<Fuerza_{i-1}*\textit{\textcolor{blue}{B\_Tolerancia\_intermedia}}
        \end{align*}
        \item Ciclo Final\\
        Se establece que el muro falla cuando 
        \begin{align*}
            Fuerza_{end}<Fuerza_{end-1}*\textit{\textcolor{blue}{B\_Tolerancia\_final}}
        \end{align*}
    \end{itemize}
    Donde $i$ es el nivel de deformación.
    Se recomienda establecer la tolerancia intermedia inferior a la tolerancia final para evitar falsos positivos de puntos de falla.
    \item Falla por Pérdida de Fuerza en mismo nivel de deformación
    Se establecieron dos posibles casos
    \begin{itemize}
        \item Ciclos intermedios\\
        Para niveles de deformación distintos al último alcanzado por el muro. Se establece que el muro falla cuando 
        \begin{align*}
            \frac{Fuerza_{i,k}}{Fuerza_{i,1}}< \textit{\textcolor{blue}{B\_tolerancia\_Perdida\_F}}
        \end{align*}
        Donde $i$ corresponde al nivel $i$ de deformación y $k$ al numero del ciclo a ese nivel de deformación 
        \item Ciclo Final\\
        Se establece que el muro falla cuando 
        \begin{align*}
            \frac{Fuerza_{i,k}}{Fuerza_{i,1}}< \textit{\textcolor{blue}{B\_tolerancia\_Perdida\_F\_final}}
        \end{align*}
        Donde $i$ corresponde al nivel $i$ de deformación y $k$ al numero del ciclo a ese nivel de deformación 
    \end{itemize}
    Se recomienda establecer la tolerancia intermedia inferior a la tolerancia final para evitar falsos positivos de puntos de falla, especialmente en protocolos con múltiples ciclos a un nivel de deformación.
\end{itemize}
A continuación se muestra un ejemplo de las detección de  puntos de falla.
   \begin{figure} [H]
    \centering
    \includegraphics[width=0.6\textwidth]{Figuras/falla.PNG}
    \caption{\label{Bi-Lineal 2}Puntos de falla}
    \end{figure}
En la figura \ref{Bi-Lineal 2} la falla detectada en la parte positiva de a histéresis corresponde a "Falla por Pérdida de Fuerza en Mismo Nivel de Deformación". En cambio, la falla en el lado negativo de la histéresis corresponde a una "Falla por Pérdida de Fuerza a Deformación Mayor". En la figura anterior, el valor de las tolerancias se consideró como 0.98 de manera de poder producir una figura que refleje ambos modos de falla en una sola figura.

\section{DataProcesada.Backbone.Ductilidad}
Corresponde a un array de dimensiones $2x1$, donde el primer elemento corresponde a la ductilidad de la parte positiva de las histéresis y el segundo elemento a la ductilidad de la parte negativa. La ductilidad se calculó como 
\begin{align*}
    \mu=\frac{d_u}{d_y}
\end{align*}
Donde $d_y$ corresponde a a deformación de fluencia del modelo bi-lineal ajustado y  $d_u$ corresponde a la deformación última. 
La convención para estimar la deformación última queda determinada por el valor de la variable \textit{\textcolor{blue}{DuctilityOption}}, específicamente 
\begin{itemize}
    \item \textit{\textcolor{blue}{DuctilityOption}}=1\\
    Se considera que la deformación última es el valor máximo del modelo bi-lineal ajustado
    \item \textit{\textcolor{blue}{DuctilityOption}}=2\\
    Si se detectó algún punto de falla, se considera que la deformación última es la deformación asociada al punto de falla. Si no se detectó un punto de falla, se utiliza la convención de \textit{\textcolor{blue}{DuctilityOption}}=1
\end{itemize}
\newpage
\chapter{DataProcesada.Otros}
El atributo \textit{DataProcesada.Otros} es a su vez un  struct con cuatro atributos 
\section{\textit{DataProcesada.Otros.Outlier}}
Es un \href{https://www.mathworks.com/help/matlab/logical-operations.html}{logical array} con los índices de la data original de los puntos identificados como \textit{outliers}
\section{\textit{DataProcesada.Otros.PuntosNegInicial}}
Corresponde  un entero que indica la cantidad de puntos que se eliminaron inicialmente pues poseían deformación o fuerza negativa, como parte de lo descrito en la sección pre-procesamiento .
\section{\textit{DataProcesada.Otros.CiclosMuyCortos}}
Corresponde a un cell, donde la posición en la primera fila corresponde al numero del ciclo detectado, y categorizado como muy corto, y la segunda fila a  un array con los  puntos que componen este ciclo.
\section{\textit{DataProcesada.Otros.Comentarios}}
Corresponde a un string con el comentario ingresado de en la sección de carga de data experimental.
\newpage









\chapter{Anexo A. Funciones Auxiliares}
\section*{Validación Visual}
La validación visual consiste de gráficos que muestras los cambios a la data experimental y los principales resultados. A continuación, se muestran la funciones que realizan esos gráficos sus resultados.\\

 \textit{OutliersGrafico(RawData,Data,oulierindex)}\\
    Esta función gráfica la data original y la data sin \textit{outliers}. Los inputs de esta función son variables intermedias que se utilizan dentro de la rutina principal.\\
    
    \begin{figure} [H]
    \centering
    \includegraphics[width=\textwidth]{Figuras/outliedetectados.png}
    \caption{\label{prepro} Resultados de detección de \textit{outliers}}
\end{figure}
\textit{GraficarHisteresis(DataProcesada,n)}\\
Esta función recibe el \textit{struct} final y un entero, a continuación se presentan los posibles valores del entero y su significado
\begin{itemize}
    \item $n=1$\\
    Se grafica la histéresis con a data procesada, la envolvente el modelo bilineal ajustado.\par 
   En la figura \ref{prepro3} se muestra el resultado de la función con $n=1$ 
     \begin{figure} [h!]
    \centering
    \includegraphics[width=0.8\textwidth]{Figuras/Figuras Introducción/Resuemne_total.png}
    \caption{\label{prepro3}Validación visual de backbone y bilineal}
    \end{figure}
    \item $n=2$\\
    Se grafican todos los ciclos identificados. Se recomienda tener cuidado con esta opción , pues dependiendo del las características del computador, esto puede demorar múltiples minutos  memoria para  en ensayos con muchos ciclos($ ~k>50$). En la figura \ref{prepro4} se muestra el resultado de la función con $n=2$
    \begin{figure} [h!]
    \centering
    \includegraphics[width=1\textwidth]{Figuras/Consola/Ejemplo por ciclo.png}
    \caption{\label{prepro4}Validación visual de ciclos identificados}
    \end{figure}
    \item $n=3$\\
    Se grafican todos los ciclos identificados, las rigideces secantes y todas las rigideces de la parte positiva. Similar a la opción anterior, en ensayos con mucho ciclos, Matlab puede demorarse en crear todos los gráficos. En la figura \ref{prepro5} se muestra el resultado de la función con $n=3$
    \begin{figure} [h!]
    \centering
    \includegraphics[width=1\textwidth]{Figuras/Consola/Ejemplo ciclo con rigidez.png}
    \caption{\label{prepro5}Validación visual de ciclos y rigideces.}
    \end{figure}
\end{itemize}
 \textit{GraficarCiclos(DataProcesada)}\\
La función grafica los ciclos detectados y la data experimental en una figura, de manera de poder evaluar rápidamente el procesamiento realizado por el script. En al figura \ref{prepro6} se muestran con distintos colores los ciclos detectados
\begin{figure} [h!]
    \centering
    \includegraphics[width=0.8\textwidth]{Figuras/validar ciclos.png}
    \caption{\label{prepro6} Validación visual de  detección de ciclo}
    \end{figure}
\newpage\null\newpage
\textit{Resultados.m}\\
Esta función gráfica los principales resultados del análisis realizado por el script \textit{Analisis\_Data\_Experimental.m}, en particular, la pérdida de rigidez en función de la deriva y  de la energía disipada, como se observa en la figuras \ref{D_K_deriva} y \ref{D_K_energia}.
\begin{figure} [h!]
    \centering
    \includegraphics[width=1\textwidth]{Figuras/DRigi_deriva.png}
    \caption{\label{D_K_deriva} Pérdida de rigidez en función de deriva}
    \end{figure}

\begin{figure} [h!]
    \centering
    \includegraphics[width=1\textwidth]{Figuras/Drigi_energia.png}
    \caption{\label{D_K_energia} Pérdida de rigidez en función de deriva}
    \end{figure}

\end{document}
